\documentclass[12pt]{article}
\usepackage{graphicx}
\usepackage{ragged2e}
\usepackage{array}
\usepackage{amsmath,xparse}
\usepackage{longtable}
\pagestyle{empty}
\newcolumntype{L}[1]{>{\raggedright\let\newline\\\arraybackslash\hspace{0pt}}m{#1}}
\newcolumntype{C}[1]{>{\centering\let\newline\\\arraybackslash\hspace{0pt}}m{#1}}
\newcolumntype{R}[1]{>{\raggedleft\let\newline\\\arraybackslash\hspace{0pt}}m{#1}}
\begin{document}
	\centering{\bf{KOLHAPUR INSTITUTE OF TECHNOLOGY'S}}\par
	{\bf{COLLEGE OF ENGINEERING (AUTONOMOUS),KOLHAPUR}}
	\par\noindent\rule{\textwidth}{0.4pt}
	
	\centering{\bf{FIRST YEAR B.Tech}}\par
	\centering{\bf{MID SEMESTER EXAMINATION}}\par
	\centering{\bf{Mathematics2 (BSM2)}}\par
	\begin{flushleft}
		Day and Date :{}\hspace{5.5cm}PRN:
	\end{flushleft}
	
	\begin{flushleft}
		Time :{}\hspace{7cm}Max Marks:{30}\\
	\end{flushleft}
	\noindent\rule{\textwidth}{0.1pt}
\begin{flushleft}
	{\bf Instructions:}\\
	{\hspace{0.5cm} \bf IMP: Verify that you have received question paper with correct course, code, branch, etc}\\
	\hspace{1cm}i) All Questions are Compulsory\\
	\hspace{1cm}ii)Figure to right indicate full marks\\
	\hspace{1cm}iii)Assume suitable data wherever necessary\\
\end{flushleft}

	\begin{flushleft}
	\bf{QNo}\hspace{1.2cm} \bf{Question} \hspace{5.5cm}  \bf{Marks} \hspace{0.2cm} \bf{CO} \hspace{0.2cm}	\bf{BL}	
	
\end{flushleft} 
	\begin{longtable}{|L{1cm}|L{8cm}|C{1cm}|C{1cm}|C{1cm}|}\hline
		\bf{1}. & \bf{Attempt} \bf2 \bf{out} of \bf3 & \bf6  & & \\ \hline
				1.A & What is the value of 3! \newline
					
		A)6\newline
		B)9\newline
		C)4\newline
		D)3 &
		4 &
		CO1&
		1 \\ \hline
		
				1.B & The probability of an event happening is 0.3.What is the probability of not happening \newline
					
		A)0,5\newline
		B)0.3\newline
		C)0.2\newline
		D)0.7 &
		4 &
		CO1&
		1 \\ \hline
		
				1.C & Area of circle can be defined as. where r is radius \newline
					
		A)2*3.14*r\newline
		B)3.14*r*r\newline
		C)3.14*r*r*3\newline
		D)4*3.14*r*r &
		4 &
		CO1&
		5 \\ \hline
		
		
	\end{longtable}

	\begin{longtable}{|L{1cm}|L{8cm}|C{1cm}|C{1cm}|C{1cm}|}\hline
	\bf2. & \bf{Attempt} \bf{2} \bf{out of} \bf{2} & \bf{12}  & & \\ \hline





		2.A &
	Find the area of rectangle from the below figure \newline
			\begin{center}
		\includegraphics[width=4cm,height=3cm]{media/diagrams/2020/09/14/rectsquare.png}\\\bf{Figure }\bf2.A		
	\end{center}
		
	 &  4 & CO1 & 5\\ \hline
		2.B &
	Derive that iternal sum of angles in an triangle get a sum of 180 degrees \newline
			\begin{center}
		\includegraphics[width=4cm,height=3cm]{media/diagrams/2020/09/14/rectsquare_k3vvwfe.png}\\\bf{Figure }\bf2.B		
	\end{center}
		
	 &  4 & CO1 & 4\\ \hline
	\end{longtable}


\begin{longtable}{|L{1cm}|L{8cm}|C{1cm}|C{1cm}|C{1cm}|}\hline
	\bf3. & \bf{Attempt} \bf{1} \bf{out of} \bf{2} & \bf{12}  & & \\ \hline





		3.A &
	Evaluate $\int x^{3}dx$ \newline
			
	 &  4 & CO2 & 3\\ \hline
		3.B &
	Evaluate $\frac{x\frac{x}{5}}{dx}$ \newline
			
	 &  4 & CO1 & 1\\ \hline
	\end{longtable}



\end{document}
	
	